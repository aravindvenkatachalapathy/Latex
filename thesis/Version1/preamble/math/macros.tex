%%% Macros for math symbols

%% Punctuation in math mode
\NewDocumentCommand{\eqend}{}{\,.}
\NewDocumentCommand{\eqcomma}{}{\,,}

%% Equals in given dimension
\NewDocumentCommand{\eqdim}{O{} m}{
    \IfBlankTF{#1}
    {\mathrel{\overset{\mathcolor{black!70}{d\.=\.#2}}{\scalebox{2.8}[1]{\(=\)}}}}
    {\mathrel{\overset{\mathcolor{black!70}{#2}}{\scalebox{#1}[1]{\(=\)}}}}
}
%% Phantom with relation spacing
\NewDocumentCommand{\relphantom}{m}{\mathrel{\phantom{#1}}}
%% Continuation of expression to the next line
\NewDocumentCommand{\graytimes}{}{\mathcolor{gray}{\times}}

% %% (Optional) Automatically use \mathcolor in math mode
% \NewCommandCopy{\textcolorOrig}{\textcolor}
% \RenewDocumentCommand{\textcolor}{O{} m m}{%
%     \ifmmode%
%         \let\colorcmd\mathcolor%
%     \else%
%         \let\colorcmd\textcolorOrig%
%     \fi%
%     \IfBlankTF{#1}{\colorcmd{#2}{#3}}{\colorcmd[#1]{#2}{#3}}%
% }

%% Determinants (use normal position of superscript/subscript)
\NewCommandCopy{\olddet}{\det}
\RenewDocumentCommand{\det}{}{\olddet\nolimits}
\DeclareMathOperator{\Det}{Det}

%% Misc math macros
\NewCommandCopy{\transpose}{\intercal}
\DeclareMathOperator{\vol}{vol}
\DeclareMathOperator{\sgn}{sgn}
\NewDocumentCommand{\Id}{}{\bm{1}}
\NewDocumentCommand{\const}{}{\text{const.}}
\DeclareMathOperator{\Span}{Span}
\DeclareMathOperator{\diag}{diag}
\NewDocumentCommand{\bdry}{}{\partial}
\NewDocumentCommand{\disjunion}{}{\sqcup} % TODO: spacing too small sometimes (in subscripts)
\NewDocumentCommand{\inclusion}{s}{\IfBooleanTF{#1}{\hookleftarrow}{\hookrightarrow}}
\NewDocumentCommand{\surjection}{}{\twoheadrightarrow}
\NewDocumentCommand{\exchange}{O{\quad}}{\xleftrightarrow{#1}}

%% Imaginary unit and Euler's number
%% inspired by Niklas Beisert
\NewDocumentCommand{\I}{}{\mathring{\imath}}
\NewDocumentCommand{\E}{s}{\IfBooleanTF{#1}{\mathrm{e}}{\mathinner{\mathrm{e}}}}

%% Big O notation (\biggO for centered O)
\NewDocumentCommand{\bigO} {l m}{\fbraces#1{\lparen}{\rparen}                   {O} {#2}}
\NewDocumentCommand{\biggO}{l m}{\fbraces#1{\lparen}{\rparen}{\raisemath{-0.1ex}{O}}{#2}}

%%% Various fields and number sets
\NewDocumentCommand{\R}{}{\mathbb{R}}
\NewDocumentCommand{\Z}{}{\mathbb{Z}}
\NewDocumentCommand{\C}{}{\mathbb{C}}
\NewDocumentCommand{\F}{}{\mathbb{F}}
\NewDocumentCommand{\N}{}{\mathbb{N}}

%% Group/Representation Theory
\DeclareMathOperator{\Hom}{Hom}
\DeclareMathOperator{\Ker}{Ker}
\DeclareMathOperator{\End}{End}
\DeclareMathOperator{\Aut}{Aut}
\DeclareMathOperator{\gl}{\mathfrak{gl}}
\DeclareMathOperator{\GL}{\mathsf{GL}}
\let\sl\relax % override (anyway deprecated) `sl` command
\DeclareMathOperator{\sl}{\mathfrak{sl}}
\DeclareMathOperator{\SL}{\mathsf{SL}}
\DeclareMathOperator{\oo}{\mathfrak{o}}
\DeclareMathOperator{\OO}{\mathsf{O}}
\DeclareMathOperator{\so}{\mathfrak{so}}
\NewDocumentCommand{\SO}{s}{\operatorname{\IfBooleanTF{#1}{\widetilde{\mathsf{SO}}}{\mathsf{SO}}}}
\DeclareMathOperator{\ISO}{\mathsf{ISO}}
\DeclareMathOperator{\spin}{\mathfrak{spin}}
\DeclareMathOperator{\Spin}{\mathsf{Spin}}
\DeclareMathOperator{\U}{\mathsf{U}}
\DeclareMathOperator{\su}{\mathfrak{su}}
\DeclareMathOperator{\SU}{\mathsf{SU}}
\DeclareMathOperator{\Sp}{\mathsf{Sp}}
\NewDocumentCommand{\g}{}{\mathfrak{g}}
\NewDocumentCommand{\rankgroup}{}{\mathsf{r}}

%%% Differential Geometry

\DeclareMathOperator{\Sect}{Sect}
\NewDocumentCommand{\Tangent}{s e{_}}{\IfBooleanTF{#1}{\bm{T}^{*}\mspace{-2mu}}{\bm{T}}\IfValueT{#2}{_{\mspace{-2mu}#2}}}
\NewDocumentCommand{\TSect}{s e{^_}}{\IfBooleanTF{#1}{\mathcal{T}^{*}\mspace{-2mu}}{\mathcal{T}}\IfValueT{#2}{^{#2}}\IfValueT{#3}{_{\mspace{0mu}#3}}\mspace{-2mu}}
\NewDocumentCommand{\Lie}{}{\mathcal{L}}
\DeclarePairedDelimiterX{\Liebracket}[2]{\dblbracketleft}{\dblbracketright}{#1,#2}
% \DeclarePairedDelimiterX{\Liebracket}[2]{\lbrack}{\rbrack}{#1,#2}

\DeclareMathOperator{\Tor}{\mathsf{Tor}}
\NewDocumentCommand{\Riem}{O{}}{\IfBlankTF{#1}{\bm{R}}{\tens{\bm{R}}{#1}}}
\NewDocumentCommand{\Ric}{O{}}{\IfBlankTF{#1}{\bmrm{Ric}}{\tens{\bmrm{Ric}}{#1}}}
\NewDocumentCommand{\Rscalar}{}{\mathcal{R}}

%% Override `physics` package command `\dd` to optionally use bold `d`
\RenewDocumentCommand{\dd}{s O{} m}{
    \mathinner{
        \IfBooleanTF{#1}{\bm{\mathrm{d}}}{\mathrm{d}}^{\mspace{-1mu}#2}#3
    }
}
%% Bold differential
\NewCommandCopy{\dotunderaccent}{\d}
\RenewDocumentCommand{\d}{O{} m}{\TextOrMath{\dotunderaccent{#2}}{\dd*[#1]{#2}}}


%% Covariant derivative (with optional arguments for space tweaking)
\NewDocumentCommand{\cder}{s O{a} e{_}}{
    \IfBooleanT{#1}{\mspace{-2mu}} % use starred variant when more \cder after each other
    \bm{\nabla}
    \IfValueT{#3}{             % when subscript index is given
        _{                     % start subscript
            \mspace{-7mu}      % remove extra space after nabla
            \chphantom{#3}{#2} % print #3 with the width of #2
        }
    }
}
%% Coordinate derivative
\NewDocumentCommand{\del}{s}{\IfBooleanTF{#1}{\partial}{\bm{\partial}}}
\NewDocumentCommand{\fdel}{O{} m}{{\frac{\del #1}{\del #2}}}
\NewDocumentCommand{\Dv}{O{} m}{{\frac{\bm{D} #1}{d #2}}}
\NewDocumentCommand{\Laplacian}{}{\triangle}
\NewDocumentCommand{\dAlembertian}{s}{
    \mathord{\raisemath{0.07ex}{
            \mspace{1.5mu}
            \IfBooleanTF{#1}{\square[0.35em][boxframe=0.05em]}{\square[][boxframe=0.06em]}
            \.\.
        }}
}
% \RenewDocumentCommand{\dAlembertian}{s}{\mathord{\raisemath{-0.05ex}{\oldsquare}}}

%%% Tweaking of math index positioning, spacing, etc.
%%% (also definitions of new arrows, etc.)

%% TODO: see if \cramped{...} could be used sometimes to make
%%       (inline) math prettier (to avoid weird line spacing weird)

%% Modify size of `\bigwedge`/`\bigodot` and the position of their superscripts
\NewDocumentCommand{\Ext}{e{^}}{
    \scalerel*{\bigwedge}{\xmathstrut[-0.1]{0.1}} % scale down the \bigwedge a bit
    \IfValueT{#1}{^{\mspace{-2mu}#1}} % shift possible superscript little bit to the left
}
\NewDocumentCommand{\Odot}{e{^}}{
    \scalerel*{\bigodot}{\xmathstrut[-0.1]{0.1}} % scale down the \bigodot a bit
    \IfValueT{#1}{^{\mspace{-0mu}#1}} % (do not) shift possible superscript little bit to the left
}

%% Exchange \epsilon and \varepsilon
\NewCommandCopy{\oldepsilon}{\epsilon}
\RenewCommandCopy{\epsilon}{\varepsilon}
\RenewCommandCopy{\varepsilon}{\oldepsilon}

%% Math version of `\raisebox` and `\scalebox`
\NewDocumentCommand{\raisemath}{m}{\mathpalette{\raisemathAux{#1}}}% \raisemath{<len>}{...}
\NewDocumentCommand{\raisemathAux}{mmm}{\raisebox{#1}{\(#2#3\)}}
\NewDocumentCommand{\scalemath}{m O{}}{\mathpalette{\scalemathAux{#1}[#2]}}% \scalemath{<factor>}{...}
\NewDocumentCommand{\scalemathAux}{m O{} mm}{\IfBlankTF{#2}{\scalebox{#1}{\(#3#4\)}}{\scalebox{#1}[#2]{\(#3#4\)}}}

%% Modify subscript position
\NewDocumentCommand{\lowerindex}{O{0pt} O{0pt} e{_^}}{
    \IfValueT{#3}{_{\raisemath{#1}{#3}}}
    \IfValueT{#4}{^{\raisemath{#2}{#4}}}
}

%% Modify superscript/subscript positions for some greek letters
\NewCommandCopy{\oldchi}{\chi}
\RenewDocumentCommand{\chi}{}{\oldchi\lowerindex[-1.5pt]}
\NewCommandCopy{\olddelta}{\delta}
\RenewDocumentCommand{\delta}{}{\olddelta\lowerindex[1pt][1.0pt]}
\NewDocumentCommand{\bmdelta}{}{\bm{\olddelta}\lowerindex[1pt][1.0pt]}
\NewDocumentCommand{\altdelta}{}{{\olddelta\xmathstrut[-0.2]{0.0}}}
\NewDocumentCommand{\altbmdelta}{}{{\bm{\olddelta}\xmathstrut[-0.2]{0.0}}}

%% Smaller \in, \notin, \subset, ...
%% https://tex.stackexchange.com/questions/34393/the-mysteries-of-mathpalette
\NewDocumentCommand{\smallermath}{O{0.05ex} O{0.85} m}{\raisemath{#1}{\scalemath{#2}{#3}}}
\NewDocumentCommand{\smallplus}  {}{\mathrel{\smallermath{+}}}
\NewDocumentCommand{\textplus}   {}{\ensuremath{\.\smallplus\.}}
\NewDocumentCommand{\smallin}    {}{\mathrel{\smallermath{\in}}}
\NewDocumentCommand{\smallnotin} {}{\mathrel{\smallermath{\notin}}}
\NewDocumentCommand{\smallsubset}{}{\mathrel{\smallermath{\subset}}}
\NewDocumentCommand{\smallotimes}{}{\mathbin{\smallermath[0ex]{\oldotimes}}}
% \NewDocumentCommand{\smallotimes}{}{\.\smallerrel{\oldotimes}\.}


%% Fix spacing of \left( .. \middle| .. \right)
\NewCommandCopy{\leftOrig}{\left}
\NewCommandCopy{\rightOrig}{\right}
\NewCommandCopy{\middleOrig}{\middle}
\RenewDocumentCommand{\left}{}{\mathopen{}\mathclose\bgroup\leftOrig}
\RenewDocumentCommand{\right}{}{\aftergroup\egroup\rightOrig}
\RenewDocumentCommand{\middle}{sm}{\IfBooleanTF{#1}{\.\middleOrig#2\.}{\mathrel{}\middleOrig#2\mathrel{}}}
\NewDocumentCommand{\innermiddle}{m}{\mathinner{}\middleOrig#1\mathinner{}}

%% Less space around \setminus (\mathord instead of \mathinner)
\DeclareMathSymbol{\setminus}{\mathord}{symbols}{"6E}

%% Redefine `\square` to Young tableaux
\usepackage{ytableau}           % Young Tableaux
\ytableausetup{centertableaux}
\NewCommandCopy{\oldsquare}{\square}
\NewCommandCopy{\oldotimes}{\otimes}
\RenewDocumentCommand{\square}{O{0.55em} O{} O{}}{%
    \IfBlankTF{#1}{\ytableausetup{boxsize=0.55em}}{\ytableausetup{boxsize=#1}}%
    \ytableausetup{aligntableaux=top, boxframe = normal, #2}%
    \operatorname{\ydiagram[#3]{1}}%
}
\NewDocumentCommand{\smallsquare}{O{0.35em}}{\square[#1]}

%% Sizeable bullet
\NewDocumentCommand{\sbullet}{O{0.5}}{%
    \mathbin{\ThisStyle{\vcenter{\hbox{\scalebox{#1}{\(\SavedStyle\bullet\)}}}}}
}
\NewDocumentCommand{\fdot}{}{\mathcolor{black!80}{\sbullet[0.65]}}
\NewDocumentCommand{\adot}{}{\sbullet[0.52]}
\NewDocumentCommand{\idot}{s}{\mathcolor{darkgray}{\IfBooleanTF{#1}{\.\sbullet[0.4]\.}{\sbullet[0.4]}}}

%% (abstract) index placeholders
\NewDocumentCommand{\aind}{}{\bullet}
\NewDocumentCommand{\dind}{}{\mathcolor{black!60}{\sbullet[1.2]}}
\NewDocumentCommand{\ind}{}{\mathcolor{lightgray}{\bullet}}

%% Argument placeholder
\NewDocumentCommand{\argument}{s O{} O{1}}{%
    \def\squaresize{1.0}%                   % default size
    \IfBooleanT{#1}{\def\squaresize{0.7}}%  % if starred, set size corresponding to scriptstyle
    \IfBlankF{#2}{\def\squaresize{#2}}%     % if optional argument is passed, set size to custom value
    % TODO: utilize \mathpalette
    \scalebox{\squaresize}{%
        \begin{tikzpicture}[baseline=-#3*0.6ex]
            \node(char)[draw, shape=rectangle, dash=on 1.2pt off 1.05pt phase 0.5pt, dash expand off,
                inner ysep=2pt, inner xsep=2pt, minimum size=0.6em, rounded corners=2pt] {};
            %% alternative:
            % \node(char)[draw, shape=rectangle, dash=on 1.1pt off 0.8pt phase 0.5pt, dash expand off,
            %     inner ysep=2pt, inner xsep=2pt, minimum size=0.6em, rounded corners=2pt] {};
        \end{tikzpicture}%
    }%
}

%% Curly arrows
\tikzset{
    leadsto/.style={-{Stealth[length=0.6em,open,round]},decorate,decoration={snake,amplitude=0.20ex,segment length=0.5em,pre length=0.2em,post length=0.6em}},
    toleads/.style={{Stealth[length=0.6em,open,round]}-,decorate,decoration={snake,amplitude=0.20ex,segment length=0.5em,pre length=0.6em,post length=0.2em}},
    correspondsto/.style={{Stealth[length=0.6em,open,round]}-{Stealth[length=0.6em,open,round]},decorate,decoration={snake,amplitude=0.20ex,segment length=0.5em,pre length=0.7em,post length=0.7em}},
}
\NewDocumentCommand{\longleadsto}{s O{} O{}}{%
    \ensuremath{\mathrel{
            \tikz[baseline=-0.5ex, inner sep=0ex, font=\scriptsize]{
                \node[minimum width=2.15em, inner xsep=0.6em, align=center] (a){\hphantom{#2}\\[0ex]\hphantom{#3}};
                \IfBlankF{#2}{\node[inner sep=0.3ex, above=0.3ex, xshift=-0.1em] at (a){#2};}
                \IfBlankF{#3}{\node[inner sep=0.3ex, below=0.3ex, xshift=-0.1em] at (a){#3};}
                \IfBooleanTF{#1}
                {\draw[line width=0.6pt, toleads] (a.west) -- (a.east);}
                {\draw[line width=0.6pt, leadsto] (a.west) -- (a.east);}
            }
        }}%
}
%% https://tex.stackexchange.com/questions/170092/xleftrightarrows-command-in-tikz-with-arrows-matching-the-latex-font?rq=1
\NewDocumentCommand{\correspondsto}{O{}O{}}{%
    \ensuremath{\mathrel{
            \tikz[baseline=-0.5ex, inner sep=0ex, font=\scriptsize]{
                \node[minimum width=3.48em, inner xsep=0.8em, align=center] (a){\hphantom{#1}\\[0ex]\hphantom{#2}};
                \IfBlankF{#1}{\node[inner sep=0.3ex, above=0.3ex] at (a){#1};}
                \IfBlankF{#2}{\node[inner sep=0.3ex, below=0.3ex] at (a){#2};}
                \draw[line width=0.6pt, correspondsto] (a.west) -- (a.east);
            }
        }}%
}

%% Quotient macro
\NewDocumentCommand{\bigslant}{O{0.2}O{1.7}mm}{
    \left.\mspace{-1mu}\raisemath{#1em}{#3}
    \mspace{-#2mu} \middleOrig/ \mspace{-\fpeval{#2+1}mu}
    \raisemath{-#1em}{#4} \mspace{-\fpeval{5*#1}mu} \right.
}
% \NewDocumentCommand{\coset}{O{0.05} O{0.1} m m}{\bigslant[#1][#2]{#3}{#4}}
\NewDocumentCommand{\coset}{O{}O{} m m}{#3/#4}
\NewDocumentCommand{\gt}{}{\bigslant{\g}{\mathfrak{t}}}
\NewDocumentCommand{\gts}{}{\bigslant[0.15]{\scriptstyle\g}{\scriptstyle\mathfrak{t}}}
\NewDocumentCommand{\onehalf}{}{\mspace{-2mu}\bigslant[0.15]{\scriptscriptstyle 1 \mspace{-1.2mu}}{\scriptscriptstyle 2}}

%% Cancel macro
\definecolor{cancelgray}{gray}{0.85}
\tikzset{
    main node/.style={inner sep=0,outer sep=0},
    label node/.style={inner sep=0.3em,font=\tiny,overlay},
    strike out/.style={shorten <=-.2em,shorten >=-.2em,overlay,thick,double distance = 0em,line cap=round},
}
\NewDocumentCommand{\cancel}{O{cancelgray}mo}{%
    \tikz[baseline=(N.base)]{
        \node[main node](N){\(#2\)};
        \IfValueT{#3}{\node[label node, gray, anchor=south] at (N.north){#3};}
        \draw[strike out, #1]  (N.south west) -- (N.north east);
    }
}
