%% Configuration of figures, tables, captions, ...

%% Use same numbering for figures, tables, and equations
\makeatletter
\let\c@figure\c@table
\let\c@equation\c@table
\makeatother

%%% Graphics
\usepackage{graphicx}   % embedding of pictures
\graphicspath{          % default paths to figures
    {./figures/}
    {./figures/Inkscape/}
    {./frontmatter/img/}
}
%% Macro for appending to the graphics path
\ExplSyntaxOn
\NewDocumentCommand{\appendtographicspath}{m}{
    \tl_if_exist:cF { Ginput@path } { \tl_new:c { Ginput@path } }
    \tl_gput_right:cn {Ginput@path} { #1 }
}
\ExplSyntaxOff

%%% Tables
\usepackage{adjustbox}      % center big tables
\usepackage{array}          % custom column types in tables
\usepackage{booktabs}       % improved horizontal lines in tables

%% Increase default vertical space between rows in tables (default is 1.0)
\renewcommand*{\arraystretch}{1.1}

% HACK: `ninecolors` is needed for `tabularray`, but fails to load with
%       rgb color model -> see https://tex.stackexchange.com/a/614702
\selectcolormodel{natural}  % temporarily switch to natural color model
\usepackage{ninecolors}     % now we can load `ninecolors` package
\selectcolormodel{rgb}      % switch back to RGB color model

\usepackage{tabularray}     % advanced LaTeX tables
\usepackage{codehigh}       % verbatim in tables (with `\fakeverb` macro)
\UseTblrLibrary{amsmath, booktabs, siunitx} % load libraries for `tabularray`


%% Does \centering automatically, provides side captions (`\fcapside`) and much more
%% Inspired by https://collaborating.tuhh.de/m21/public/theses/itt-latex-template
\usepackage{floatrow}
\floatsetup{ % for all floats
    footnoterule = none,
    footposition = bottom,
}
\floatsetup[figure]{
    capbesideposition = right,
    capbesidesep = quad,
}

%% If you want to position the caption above the figure, use the following
% \floatsetup[table]{
%     style = plaintop, % caption always above, no matter where \caption is called
% }

%% Set caption width to be the same as the table width
% \floatsetup[longtable]{LTcapwidth=table} % https://tex.stackexchange.com/a/345772/120853

\usepackage{caption}    % customizing captions in floating environments
\usepackage{subcaption}

% \DeclareCaptionLabelSeparator{slash}{~/~} % `␣/␣` between label and caption
\DeclareCaptionLabelSeparator{slash}{\hspace{0.25em}/\hspace{0.25em}} % `␣/␣` between label and caption
\captionsetup{
    format        = plain,  % no hanging indent
    indention     = 0.6em,  % but still slightly indent the caption
    % format        = hang,   % alternative: hanging indent
    font          = small,
    labelfont     = {sf,bf},
    labelsep      = slash,
    labelformat   = simple,
    tableposition = bottom,
    parskip       = .3\baselineskip plus 1pt,
}

\makeatletter
% Make this new length and indent, same length as regular caption indent:
\newlength{\floatfootruleindent}
\setlength{\floatfootruleindent}{\caption@indent}% Set the new length
% A bit hacky; introduce a rule underneath caption if \floatfoot is called:
\renewcommand*{\floatfootskip}{2pt\color{Gray50}\hspace{\floatfootruleindent}\hrulefill}%
\makeatother

\DeclareCaptionFont{ftfont}{%
    \scriptsize%
    \color{Gray60}%
    \sffamily\raggedleft%
}
\captionsetup[floatfoot]{
    footfont=ftfont, % https://tex.stackexchange.com/q/9547/120853
}

%%% You can change the justification of all side-captions here
% \captionsetup[capbesidefigure]{
%     % When using sidecaptions, the linewidth can be rather small and awkward breaks and
%     % many underfull hboxes occur. Therefore, raggedright.
%     justification=raggedright,
% }
%
% \captionsetup[subfigure]{%
%     labelformat=simple,% 'parens' uses parantheses, 'brace' just the right one
%     labelsep=slash,%
%     labelfont={sf,bf},%
%     list=off,% list=off removes subfigures from LoF
% }%
%
% \captionsetup[subtable]{%
%     labelformat=simple,% 'parens' uses parantheses, 'brace' just the right one
%     labelsep=slash,%
%     labelfont={sf,bf},%
%     list=off,% list=off removes subfigures from LoF
% }%

%% Change counter from Arabic number to letter:
\renewcommand*{\theContinuedFloat}{\alph{ContinuedFloat}}
